\documentclass{beamer}
\usetheme{intridea}

\title{Intro to {\TeX} and {\LaTeX}}
\subtitle{The Programming Language for Creating Beautiful Documents}
\author{John McAvoy}
\institute{A-Team Workshop}
%\date{}

\begin{document}

\frame{\titlepage}

%\begin{frame}
%\frametitle{Table of Contents}
%\tableofcontents
%\end{frame}

\begin{frame}
  \frametitle{What is {\TeX} and {\LaTeX}?}

  \begin{enumerate}
    \item \textbf{\TeX} (= tau epsilon chi, and pronounced similar to "blecch",
      not to the state known for `Tex-Mex' chili) is a computer language
      designed by Donald Knuth for use in typesetting; in particular, for
      typesetting math and other technical (from Greek "techne" = art/craft, the
      stem of 'technology') material. It was t takes a "plain" text file and
      converts it into a high-quality document for printing or on-screen
      viewing.\cite{tug} \cite{wikibooks}

    \item \textbf{\LaTeX} is a macro system built on top of TeX that aims
      to simplify its use and automate many common formatting tasks. It is the
      de-facto standard for academic journals and books, and provides some of
      the best typography free software has to offer. \cite{wikibooks}

  \end{enumerate}
\end{frame}

\begin{frame}
  \frametitle{Works Cited}
  \begin{thebibliography}{1}
    \bibitem{tug} {\TeX} User Group \url{http://tug.org}
    \bibitem{wikibooks} WikiBooks:{\LaTeX} \url{https://en.wikibooks.org/wiki/LaTeX}
  \end{thebibliography}
\end{frame}

\end{document}
